\documentclass{article}
\usepackage[T1]{fontenc}
\usepackage{times}
\usepackage[english]{babel}
\usepackage[utf8x]{inputenc}

\begin{document}
\title{PyTacToe Documentation}
\author{Flavien Chamay}
\date{\today}

\maketitle

\section{Rules of Tic-Tac-Toe}
\begin{itemize}
\item We play on a grid that's 3 squares by 3 squares.
\item One player is symbolized by an "X", the other by an "O".
\item When one player has his 3 marks in a row, whether its diagonally, up-down or in a line.
\item If no player has 3 marks in a row and all 9 squares are full then the game ends.
\end{itemize}

\section{Multiplayer}
\subsection{Player VS computer}
The player is able to play against the computer. A notification will signify when the player should play.
\subsection{Player VS player}
Each player will play again each other, a notification will signify if player1 should play or player2 should play.

\section{GUI}
For the moment we will use CLI to interact with the game. A graphical interface will be implemented in the future with PyQt.

\section{AI in PyTacToe}
We will try to implement in the future an AI with degrees of difficulty: normal and unbeatable.

\end{document}
